% Template for documenting your Arduino projects
% Author:   Luis José Salazar-Serrano
%           totesalaz@gmail.com / luis-jose.salazar@icfo.es
%           http://opensourcelab.salazarserrano.com
%%% Template based in the template created by Karol Kozioł (mail@karol-koziol.net)
\def \TITLE     {Installasi Code::Blocks}
\def \AUTHOR    {Suka Isnaini, COHERENCE, Kenzanin@gmail.com}
\def \SUBJECT   {ESP8266}
\def \KEYWORDS  {Python, Python-PIP, PlatformIO}

\documentclass[a4paper,11pt]{article}

\usepackage[T1]{fontenc}
\usepackage[utf8]{inputenc}
\usepackage{graphicx}
\usepackage{xcolor}

\renewcommand\familydefault{\sfdefault}
\usepackage{tgheros}
\usepackage[defaultmono]{droidmono}

\usepackage{amsmath,amssymb,amsthm,textcomp}
\usepackage{enumerate}
\usepackage{multicol}
\usepackage{tikz}
\usepackage{courier}

\usepackage{geometry}
\geometry{total={210mm,297mm},
left=25mm,right=25mm,%
bindingoffset=0mm, top=20mm,bottom=20mm}

%\linespread{1.3}
\newcommand{\linia}{\rule{\linewidth}{0.5pt}}

% my own titles
\makeatletter
\renewcommand{\maketitle}{
\begin{center}
\vspace{2ex}
{\huge \textsc{\@title}}
\vspace{1ex}
\\
\linia\\
\@author \hfill \@date
\vspace{4ex}
\end{center}
}
\makeatother
%%%

% custom footers and headers
\usepackage{fancyhdr}
\pagestyle{fancy}
\lhead{}
\chead{}
\rhead{}
\lfoot{\TITLE}
\cfoot{}
\rfoot{Page \thepage}
\renewcommand{\headrulewidth}{0pt}
\renewcommand{\footrulewidth}{0pt}
%

% code listing settings

\usepackage{listings}

\lstset{%
  basicstyle=\footnotesize\ttfamily,       
  breakatwhitespace=false,         
  breaklines=true,                 
  captionpos=b,                   
  frame=single,                    
  keepspaces=true,                 
  tabsize=2,                       
  title=\lstname,
  emphstyle=\bfseries\color{blue}%  style for emph={} 
} 


%% language specific settings:
\definecolor{dkgreen}{rgb}{0,0.6,0}
\definecolor{dred}{rgb}{0.545,0,0}
\definecolor{dblue}{rgb}{0,0,0.545}
\definecolor{lgrey}{rgb}{0.9,0.9,0.9}
\definecolor{gray}{rgb}{0.4,0.4,0.4}
\definecolor{darkblue}{rgb}{0.0,0.0,0.6}
\lstdefinestyle{C++}{%
    language = C++,
    morecomment=[l]{//},%             treat // as comments
    morecomment=[s]{/*}{*/},%         define /* ... */ comments
    emph={HIGH, OUTPUT, LOW},%        keywords to emphasize
    backgroundcolor=\color{lgrey},  
    basicstyle=\footnotesize \ttfamily \color{black} \bfseries,   
    commentstyle=\color{dkgreen},   
    deletekeywords={...},          
    escapeinside={\%*}{*)},                  
    keywordstyle=\color{red},  
    morekeywords={}, 
    identifierstyle=\color{black},
    stringstyle=\color{blue},      
    numbers=left,                 
    numbersep=5pt,                  
    numberstyle=\tiny\color{black}, 
    rulecolor=\color{black},        
    showspaces=false,               
    showstringspaces=false,        
    showtabs=false,                
    stepnumber=1,                   
}

\lstdefinestyle{bash}{%
    language=bash,
    backgroundcolor=\color{lgrey},  
    basicstyle=\footnotesize \ttfamily \color{black} \bfseries,   
    commentstyle=\color{dkgreen},   
    deletekeywords={...},          
    keywordstyle=\color{red},  
    morekeywords={mkdir, pio}, 
    identifierstyle=\color{black},
    stringstyle=\color{blue},      
    numbers=left,                 
    numbersep=5pt,                  
    numberstyle=\tiny\color{purple}, 
    rulecolor=\color{black},        
    showspaces=false,               
    showstringspaces=false,        
    showtabs=false,                
    stepnumber=1,                       
}

% tambahan
\usepackage{inconsolata}
\usepackage{svg}
\usepackage{hyperref}
\hypersetup{colorlinks=true,allcolors=blue}
\usepackage{hypcap}
\hypersetup{
    pdftitle={\TITLE},
    pdfauthor={\AUTHOR},
    pdfsubject={\SUBJECT},
    pdfkeywords={\KEYWORDS},
    bookmarksnumbered=true,     
    bookmarksopen=true,         
    bookmarksopenlevel=1,       
    colorlinks=true,            
    pdfstartview=Fit,           
    pdfpagemode=UseOutlines,    % this is the option you were lookin for
}
\usepackage{datetime}
%%%----------%%%----------%%%----------%%%----------%%%


\begin{document}

\title{\TITLE}

\author{\AUTHOR}

\date{\today}

\maketitle
\tableofcontents
\newpage

\section{Tujuan Percobaan}
\begin{itemize}
\item Installasi Codeblocks di linux (Debian )
\item Installasi Codeblocks di windows (win7 32bit ultimate)
\end{itemize}

\section{Hasil yang diharapkan}
Applikasi codebloks terinstall dan bisa dijalankan di PC.

\section{Komponen yang digunakan}
\begin{itemize}
\item Untuk Linux 
\subitem internet connections
\subitem root access
\item Untuk Windows
\subitem Internet connections
\subitem Internet browser
\subitem Download manager jika ada
\end{itemize}

\section{Warning}
Pastikan quota internet anda berada diatas 50MB untuk windwos dan 150MB untuk linux.

\section{Langkah Percobaan}
\begin{itemize}
\item Download Codeblocks
\item Install Codeblocks
\item Jalankan Codeblocks
\end{itemize}
\subsection{Download Codeblocks}
Untuk Debian Codeblocks sudah tersedia direpository. namun saya sarankan untuk mengupdate package list dengan perintah 
\begin{lstlisting}[style=bash]
$ su
# apt-get update
\end{lstlisting}
Untuk windows, Codeblocks harus didownload terlebih dahulu melalui situs
\newline
http://www.FIXME.org/
\subsection{Install Codeblocks}
Untuk Linux Codeblocks sudah tersedia di repository masing masing dan untuk menginstallnya cukup dengan perintah 
\begin{lstlisting}[label=install-cb-apt-get,caption=Debian Install Codeblocks,style=bash]
$ su
# apt-get install codeblocks codeblocks-contrib
\end{lstlisting}
Tekan Y (Yes) jika diminta dan enter maka Debian akan segera mendownload codeblock dari repository dan menginstall nya.
\newline
Untuk windows install Codeblock seperti applikasi lainnya.
\subsection{Jalankan Codeblocks}
Pada Debian setelah terinstall kamu dapat menggunakan codeblocks dengan mengetik codeblock di terminal dan tekan enter.
\newline
Pada Windows codeblocks yang terinstall dapat dibuka di start menu.

\section{Catatan}
Codeblocks merupakan sebuah IDE yang sangat lengkap untuk mendevelop project yang menggunakan bahasa pemograman C atau C++, Pada buku ini kita tidak menggunakan project generator yang terdapat pada Codeblocks namun pembuatan project dilakukan melalui PlatformIO (pio).

\section{Pengembangan dan latihan}
Coba buat project baru dengan codeblocks, ada berbagai macam project yang bisa digenerate oleh codeblocks pelajari dan siapa tahu ada banyak hal yang bisa kamu lakukan hehe.

\end{document}
